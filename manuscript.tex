%% This is file `elsarticle-template-1-num.tex',
%%
%% Copyright 2009 Elsevier Ltd
%%
%% This file is part of the 'Elsarticle Bundle'.
%% ---------------------------------------------
%%
%% It may be distributed under the conditions of the LaTeX Project Public
%% License, either version 1.2 of this license or (at your option) any
%% later version.  The latest version of this license is in
%%    http://www.latex-project.org/lppl.txt
%% and version 1.2 or later is part of all distributions of LaTeX
%% version 1999/12/01 or later.
%%
%% The list of all files belonging to the 'Elsarticle Bundle' is
%% given in the file `manifest.txt'.
%%
%% Template article for Elsevier's document class `elsarticle'
%% with numbered style bibliographic references
%%
%% $Id: elsarticle-template-1-num.tex 149 2009-10-08 05:01:15Z rishi $
%% $URL: http://lenova.river-valley.com/svn/elsbst/trunk/elsarticle-template-1-num.tex $
%%

\documentclass[preprint,authoryear,review,12pt]{elsarticle}
%\documentclass[final,5p,times,twocolumn]{elsarticle}

%% Use the option review to obtain double line spacing
%% \documentclass[preprint,review,12pt]{elsarticle}

%% Use the options 1p,two column; 3p; 3p,twocolumn; 5p; or 5p,twocolumn
%% for a journal layout:
%% \documentclass[final,1p,times]{elsarticle}
%% \documentclass[final,1p,times,twocolumn]{elsarticle}
%% \documentclass[final,3p,times]{elsarticle}
%% \documentclass[final,3p,times,twocolumn]{elsarticle}
%% \documentclass[final,5p,times]{elsarticle}
%% \documentclass[final,5p,times,twocolumn]{elsarticle}


\usepackage{color}
\usepackage{multirow,booktabs,ctable,array}
\usepackage{lscape}
\usepackage{amsmath}
\usepackage{lineno}
\usepackage{ulem}
\usepackage{setspace}
\usepackage{listings}
\usepackage{float}


\floatstyle{plain}
\newfloat{command}{thp}{lop}
\floatname{command}{Command}

%\usepackage[nomarkers,notablist]{endfloat}

%% if you use PostScript figures in your article
%% use the graphics package for simple commands
%% \usepackage{graphics}
%% or use the graphicx package for more complicated commands
%% \usepackage{graphicx}
%% or use the epsfig package if you prefer to use the old commands
%% \usepackage{epsfig}

%% The amssymb package provides various useful mathematical symbols
\usepackage{amssymb}
%% The amsthm package provides extended theorem environments
% \usepackage{amsthm}
 
 \usepackage{makecell}

%% The lineno packages adds line numbers. Start line numbering with
%% \begin{linenumbers}, end it with \end{linenumbers}. Or switch it on
%% for the whole article with \linenumbers after \end{frontmatter}.
%% \usepackage{lineno}

%% natbib.sty is loaded by default. However, natbib options can be
%% provided with \biboptions{...} command. Following options are
%% valid:

%%   round  -  round parentheses are used (default)
%%   square -  square brackets are used   [option]
%%   curly  -  curly braces are used      {option}
%%   angle  -  angle brackets are used    <option>
%%   semicolon  -  multiple citations separated by semi-colon
%%   colon  - same as semicolon, an earlier confusion
%%   comma  -  separated by comma
%%   numbers-  selects numerical citations
%%   super  -  numerical citations as superscripts
%%   sort   -  sorts multiple citations according to order in ref. list
%%   sort&compress   -  like sort, but also compresses numerical citations
%%   compress - compresses without sorting
%%
%% \biboptions{comma,round}

% \biboptions{}

\providecommand{\OO}[1]{\operatorname{O}\bigl(#1\bigr)}

\graphicspath{{./Figures/}
                          }

\long\def\symbolfootnote[#1]#2{\begingroup%
\def\thefootnote{\fnsymbol{footnote}}\footnote[#1]{#2}\endgroup}

    \usepackage{color}

    \definecolor{listcomment}{rgb}{0.0,0.5,0.0}
    \definecolor{listkeyword}{rgb}{0.0,0.0,0.5}
    \definecolor{listnumbers}{gray}{0.65}
    \definecolor{listlightgray}{gray}{0.955}
    \definecolor{listwhite}{gray}{1.0}

\newcommand{\lstsetcpp}
{
\lstset{frame = tb,
        framerule = 0.25pt,
        float,
        fontadjust,
        backgroundcolor={\color{listlightgray}},
        basicstyle = {\ttfamily\scriptsize},
        keywordstyle = {\ttfamily\color{listkeyword}\textbf},
        identifierstyle = {\ttfamily},
        commentstyle = {\ttfamily\color{listcomment}\textit},
        stringstyle = {\ttfamily},
        showstringspaces = false,
        showtabs = false,
        numbers = none,
        numbersep = 6pt,
        numberstyle={\ttfamily\color{listnumbers}},
        tabsize = 2,
        language=[ANSI]C++,
        floatplacement=!h,
        caption={},
        captionpos=b,
        }
}


\journal{Neuroimage}

\begin{document}


\begin{frontmatter}

\title{Neuroanatomical priors improve image-based cortical thickness computation: ROI-DiReCT}



%\author[label1]{Nicholas J.~Tustison\fnref{label0}}
%%  \ead{ntustison@virginia.edu}
%  \fntext[label0]{\scriptsize Corresponding author:  PO Box 801339, Charlottesville, VA 22908; T:  434-924-7730; email address:  ntustison@virginia.edu }
%\author[label2]{Brian B.~Avants}
%\author[label2]{Philip A.~Cook}
%\author[label3]{Junghoon Kim}
%\author[label3]{John Whyte}
%\author[label2]{James C.~Gee}
%\author[label1]{James R.~Stone}
%
%\address[label1]{Department of Radiology and Medical Imaging, University of Virginia, Charlottesville, VA}
%\address[label2]{Penn Image Computing and Science Laboratory, University of Pennsylvania,
%                Philadelphia, PA}
%\address[label3]{Moss Rehabilitation Research Institute, Albert Einstein Healthcare Network, Philadelphia, PA}



%\maketitle

%\linenumbers


\begin{abstract} 
abstract
\end{abstract}

\begin{keyword}
%% keywords here, in the form: keyword \sep keyword
\end{keyword}

\end{frontmatter}
%
%
\newpage


%% MSC codes here, in the form: \MSC code \sep code
%% or \MSC[2008] code \sep code (2000 is the default)

%%
%% Start line numbering here if you want
%%
% \linenumbers

%% main text

\section*{Introduction}
%State the objectives of the work and provide an adequate background, avoiding a detailed literature survey or a summary of the results.

Working memory is a temporal process to maintain and manipulate information over short periods of time. WM has been shown to be impaired in many neurological and psychiatric syndromes including Schizophrenia. It was thought to be tightly associated with cognitive deficits in Schizophrenia. WM itself is a complex process which has been multidimensionally related to the psychosis.

Thus, a examination of the underlying neurological mechanism provided us with insight into the causes, progression and even treatment of schizophrenia. There has been abundant imaging studies in this area, but the differences among the tasks and material make it difficult to make a conclusion. There have been evidence of multi-modality deficits: visual spatial, visual object, verbal, and other types of working memory. One popular and original model of WM is a system encoding and maintaining modality-specific short-term memory and central executive component to manipulate the information. Some studies have found evidence for different patterns of functional connectivity during different working memory task conditions (e.g., as a function of load, stimulus type, or task phase).  Such findings suggest that functional connectivity changes could reflect differences in task engagement or responsivity of brain networks to modulation,rather than stable changes that persist across all task states. Some studies have found both impairments in the modality-specific perception component and the later manipulation component. In a recent meta analysis, the author draw attention to a consistent and restricted "core network" emerged form conjunctions across analyses of specific task designs and contrasts. This distributed network was believed to be active in WM task ignoring the task type , stimulus type and WM load and may be act as a base part in WM.

\section*{Materials and Methods}

\subsection*{Subjects}

The resting state fMRI data presented here was collected from six hospitals in China which paticipated in the Brainnetome Project for Schizophrenia. The six hospitals are  Patients and controls are matched as much as possible for age, sex, handedness, and race distributions with each site. Peking University Sixth Hospital (PKUH6); Beijing Huilongguan Hospital (HLG); Xijing Hospital (XJ); Henan Mental Hospital (HM);  Renmin Hospital of Wuhan University (RWU); and Zhumadian Psychiatric Hospital (ZMD). Henan Mental Hospital provided two distinct MRI scanners: Siemens (HMS) and General Electric (HMG), for a total of eight scanning centers. The study at each center was approved by the local ethical review board. All the participants provided written informed consent.

 All patients had a diagnosis of schizophrenia confirmed by trained psychiatrists using the Structured Clinical Interview for DSM-IV-TR Axis I Disorders (SCID-I/P). (First et al., 2002b). Exclusion criteria were a current neurologic disorder, a history of serious medical illness, substance dependence, pregnancy, electroconvulsive therapy within the last six months, or a diagnosis of any other Axis I disorder. The Positive and Negative Syndrome Scale (PANSS)  (Kay et al., 1989) was used to assess positive, negative, and general psychopathology symptoms in the patients. The healthy controls, who had no current axis I psychiatric disorders, were recruited from the local community near each center through advertisements. None of the HCs had any personal history of psychotic illness and no family history of psychosis in their first, second, or third degree relatives. All the participants were Han Chinese in origin, right-handed, and had no contraindications to MRI scanning. After extensive quality checking of the brain imaging data, 662 patients and 613 HCs were included in the analysis.
 
 
\subsection*{Data acquisition and preprocessing}
 
 
Two types of 3 T MRI scanners (four Siemens, three General Electric) were used at the participating centers. To ensure equivalent data acquisition protocol and high quality imaging data, the scanning parameters of the functional scans at each of the six centers were set up by an experienced researcher before data acquisition. An echo planar imaging sequence was used to obtain the
functional images, the parameters were as follows: 30 axial slices, TR = 2000 ms, TE = 30 ms, matrix = 64 × 64, flip angle = 90°, FOV = 240*240 mm 2, slice thickness = 4 mm, gap = 0.4 mm. A total of 250 brain volumes were collected, resulting in a total scan time of 500 s.The MRI scan sequences and parameters for each center are listed in eTable 1. To be noticed, the time point number of the images from ZMD is 180 and is different from the 240 of the other sites. We did not exclude this site since we are eager to keep as much data and sites as possible, to support the validation of the muli-site analysis.   


The iamges were preprocessed with a based in-house software: Brainnetome Toolkit which utilized Statistical Parametric Mapping SPM8 (http://www.fil.ion.ucl.ac.uk/spm). The pipeline includes the following steps: The first ten images were deleted for the signal equilibration. The remaining iamges were conducted for slice acquisition correction and head motion correction.  The fMRI data which had
less than 3.0 mm of head motion and 3.0° of angular rotation were included.  Moreover, the mean framewise displacement (FD) was computed by averaging FDi from every time point for each subject. There were no differences for the mean FD between groups
( t = 0.413, p = 0.682) (Table 1). Then the fMRI images were normalized to the standard Montreal Neurological Institute (MNI) template provided by SPM and resample to the 3-mm isotropic voxels. Artefacts due to changes in global, ventricle and white matter signals, residual motion were removed using voxel-wise regression. A temporal filter (0.01 Hz < f < 0.08 Hz) were used to reduce the low-frequency drift and physiological high frequency respiratory and cardiac noise.  Finally, the data was smoothed with an isotropic Gaussian kernel of 6 mm full-width at half-maximum. 


For a single label,
\begin{align}
  E(\phi(\mathbf{x},1)) = \int_0^1 \|\mathbf{v}(\mathbf{x},t)\|^2 dt +
      \|P_w(\mathbf{x}) -P_{wg}(\mathbf{x})\|^2
\end{align}

\begin{align}
  \sum_{n=1}^N E_{n}(\phi_n(\mathbf{x},1))
\end{align}


\section*{Discussion} 
%This should explore the significance of the results of the work, not repeat them. A combined Results and Discussion section is often appropriate. Avoid extensive citations and discussion of published literature.

\section*{Conclusions}

%% The Appendices part is started with the command \appendix;
%% appendix sections are then done as normal sections
%% \appendix

%% \section{}
%% \label{}

%% References
%%
%% Following citation commands can be used in the body text:
%% Usage of \cite is as follows:
%%   \citep{key}          ==>>  [#]
%%   \cite[chap. 2]{key} ==>>  [#, chap. 2]
%%   \citet{key}         ==>>  Author [#]

%% References with bibTeX database:

\section*{Acknowledgments}
%All visualizations were performed using ITK-SNAP%
%\footnote{
%http://www.itksnap.org/
%}
%\citep{Yushkevich2006} and
%DTI-TK.%
%\footnote{
%http://www.nitrc.org/projects/dtitk/
%}
%We also gratefully acknowledge Dr. Niels van Strien of the Norwegian University of Science and Technology
%who assisted in packaging the template construction algorithm in the very useful script \verb#buildtemplateparallel.sh#
%which is publicly available in ANTs.

\section*{References}

\bibliographystyle{elsarticle-harv}
\bibliography{references}


%% Authors are advised to submit their bibtex database files. They are
%% requested to list a bibtex style file in the manuscript if they do
%% not want to use model1-num-names.bst.

%% References without bibTeX database:

% \begin{thebibliography}{00}

%% \bibitem must have the following form:
%%   \bibitem{key}...
%%

% \bibitem{}

% \end{thebibliography}


\end{document}

%%
%% End of file `elsarticle-template-1-num.tex'.