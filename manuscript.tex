%% This is file `elsarticle-template-1-num.tex',
%%
%% Copyright 2009 Elsevier Ltd
%%
%% This file is part of the 'Elsarticle Bundle'.
%% ---------------------------------------------
%%
%% It may be distributed under the conditions of the LaTeX Project Public
%% License, either version 1.2 of this license or (at your option) any
%% later version.  The latest version of this license is in
%%    http://www.latex-project.org/lppl.txt
%% and version 1.2 or later is part of all distributions of LaTeX
%% version 1999/12/01 or later.
%%
%% The list of all files belonging to the 'Elsarticle Bundle' is
%% given in the file `manifest.txt'.
%%
%% Template article for Elsevier's document class `elsarticle'
%% with numbered style bibliographic references
%%
%% $Id: elsarticle-template-1-num.tex 149 2009-10-08 05:01:15Z rishi $
%% $URL: http://lenova.river-valley.com/svn/elsbst/trunk/elsarticle-template-1-num.tex $
%%

\documentclass[preprint,authoryear,review,12pt,times]{elsarticle}
\usepackage{geometry}
\geometry{left=2.54cm,right=2.54cm,top=2.54cm,bottom=2.54cm}
%\documentclass[final,5p,times,twocolumn]{elsarticle}

%% Use the option review to obtain double line spacing
%% \documentclass[preprint,review,12pt]{elsarticle}

%% Use the options 1p,two column; 3p; 3p,twocolumn; 5p; or 5p,twocolumn
%% for a journal layout:
%% \documentclass[final,1p,times]{elsarticle}
%% \documentclass[final,1p,times,twocolumn]{elsarticle}
%% \documentclass[final,3p,times]{elsarticle}
%% \documentclass[final,3p,times,twocolumn]{elsarticle}
%% \documentclass[final,5p,times]{elsarticle}
%% \documentclass[final,5p,times,twocolumn]{elsarticle}


\usepackage{color}
\usepackage{multirow,booktabs,ctable,array}
\usepackage{lscape}
\usepackage{amsmath}
\usepackage{lineno}
\usepackage{ulem}
\usepackage{setspace}
\usepackage{listings}
\usepackage{float}
\usepackage{url}


\floatstyle{plain}
\newfloat{command}{thp}{lop}
\floatname{command}{Command}

%\usepackage[nomarkers,notablist]{endfloat}

%% if you use PostScript figures in your article
%% use the graphics package for simple commands
%% \usepackage{graphics}
%% or use the graphicx package for more complicated commands
%% \usepackage{graphicx}
%% or use the epsfig package if you prefer to use the old commands
%% \usepackage{epsfig}

%% The amssymb package provides various useful mathematical symbols
\usepackage{amssymb}
%% The amsthm package provides extended theorem environments
% \usepackage{amsthm}
 
 \usepackage{makecell}

%% The lineno packages adds line numbers. Start line numbering with
%% \begin{linenumbers}, end it with \end{linenumbers}. Or switch it on
%% for the whole article with \linenumbers after \end{frontmatter}.
%% \usepackage{lineno}

%% natbib.sty is loaded by default. However, natbib options can be
%% provided with \biboptions{...} command. Following options are
%% valid:

%%   round  -  round parentheses are used (default)
%%   square -  square brackets are used   [option]
%%   curly  -  curly braces are used      {option}
%%   angle  -  angle brackets are used    <option>
%%   semicolon  -  multiple citations separated by semi-colon
%%   colon  - same as semicolon, an earlier confusion
%%   comma  -  separated by comma
%%   numbers-  selects numerical citations
%%   super  -  numerical citations as superscripts
%%   sort   -  sorts multiple citations according to order in ref. list
%%   sort&compress   -  like sort, but also compresses numerical citations
%%   compress - compresses without sorting
%%
%% \biboptions{comma,round}

% \biboptions{}

\providecommand{\OO}[1]{\operatorname{O}\bigl(#1\bigr)}

\graphicspath{{./Figures/}
                          }

\long\def\symbolfootnote[#1]#2{\begingroup%
\def\thefootnote{\fnsymbol{footnote}}\footnote[#1]{#2}\endgroup}

    \usepackage{color}

    \definecolor{listcomment}{rgb}{0.0,0.5,0.0}
    \definecolor{listkeyword}{rgb}{0.0,0.0,0.5}
    \definecolor{listnumbers}{gray}{0.65}
    \definecolor{listlightgray}{gray}{0.955}
    \definecolor{listwhite}{gray}{1.0}

\newcommand{\lstsetcpp}
{
\lstset{frame = tb,
        framerule = 0.25pt,
        float,
        fontadjust,
        backgroundcolor={\color{listlightgray}},
        basicstyle = {\ttfamily\scriptsize},
        keywordstyle = {\ttfamily\color{listkeyword}\textbf},
        identifierstyle = {\ttfamily},
        commentstyle = {\ttfamily\color{listcomment}\textit},
        stringstyle = {\ttfamily},
        showstringspaces = false,
        showtabs = false,
        numbers = none,
        numbersep = 6pt,
        numberstyle={\ttfamily\color{listnumbers}},
        tabsize = 2,
        language=[ANSI]C++,
        floatplacement=!h,
        caption={},
        captionpos=b,
        }
}


\journal{Neuroimage}
\linespread{2}
\begin{document}


\begin{frontmatter}

\title{Network Trait marker for working memory deficit in Schizophrenia: a multi site study
}



%\author[label1]{Nicholas J.~Tustison\fnref{label0}}
%%  \ead{ntustison@virginia.edu}
%  \fntext[label0]{\scriptsize Corresponding author:  PO Box 801339, Charlottesville, VA 22908; T:  434-924-7730; email address:  ntustison@virginia.edu }
%\author[label2]{Brian B.~Avants}
%\author[label2]{Philip A.~Cook}
%\author[label3]{Junghoon Kim}
%\author[label3]{John Whyte}
%\author[label2]{James C.~Gee}
%\author[label1]{James R.~Stone}
%
%\address[label1]{Department of Radiology and Medical Imaging, University of Virginia, Charlottesville, VA}
%\address[label2]{Penn Image Computing and Science Laboratory, University of Pennsylvania,
%                Philadelphia, PA}
%\address[label3]{Moss Rehabilitation Research Institute, Albert Einstein Healthcare Network, Philadelphia, PA}



%\maketitle

%\linenumbers


\begin{abstract} 
abstract
\end{abstract}

\begin{keyword}
%% keywords here, in the form: keyword \sep keyword
\end{keyword}

\end{frontmatter}
%
%
\newpage


%% MSC codes here, in the form: \MSC code \sep code
%% or \MSC[2008] code \sep code (2000 is the default)

%%
%% Start line numbering here if you want
%%
\pagewiselinenumbers
\linenumbers

%% main text

\section*{Introduction}
%State the objectives of the work and provide an adequate background, avoiding a detailed literature survey or a summary of the results.

Working memory is typically defined as the ability to maintain and manipulate information over short periods of time. WM has been shown to be impaired in many neurological and psychiatric syndromes including Schizophrenia. It was thought to be tightly associated with cognitive deficits in Schizophrenia (Green 2000). A large body of work has demonstrated that individuals with schizophrenia have deficits in working memory (Goldman-Rakic, 1991; Park and Holzman, 1992). These study may shed light on the treatment of Schizophrenia. However, there's still no conclusions on the pathology of WM. This may be due to the heterogeneous configurations of tasks and the small sample sizes of the most imaging studies. To overcome this, we assess the resting connectivity of core WM netwroks in a muti-site study framework. Such experiment configuration would help to find stable trait impairment in Schizophrenia, desregarding their age,sex, disease progress etc. This study may provide us with insight into the causes, progression and treatment of Schizophrenia. WM itself is a complex process which has been multidimensionally related to the psychosis. Thus, an examination of the underlying neurological mechanism provided us with insight into the causes, progression and even treatment of schizophrenia. 

There has been abundant imaging studies on WM deficits in Schizophrenia, but the differences among the tasks and material make it difficult to make a conclusion.  Some studies have found evidence for different patterns of functional connectivity during different working memory task conditions (e.g., as a function of load, stimulus type, or task phase). Such findings suggest that functional connectivity changes could reflect differences in task engagement or responsivity of brain networks to modulation,rather than stable changes that persist across all task states. On the other hand, working memory itself is a complex system which consist of different components. It is widely agreed that WM involves several different component processes. One popular and original model of WM is a system encoding and maintaining modality-specific short-term memory and central executive component to manipulate the information. Some studies have found both impairments in the modality-specific perception component and the later manipulation component. There have been evidence of multi-modality deficits: visual spatial, visual object, verbal, and other types of working memory. In a recent meta analysis, the author draw attention to a consistent and restricted "core network" emerged from conjunctions across analyses of specific task designs and contrasts. This distributed network was believed to be active in WM task ignoring the task type , stimulus type and WM load and may be act as a base part in WM. We restrict our analysis in these core networks.

Schizophrenia has often been conceived as a disorder of connectivity between components of large-scale brain networks(Lynall, 2010). A growing number of studies have reported altered functional connectivity in schizophrenia during putatively “task-free” states and during the performance of cognitive tasks. (Grega Repovs 2012)  A few studies has focused on the relationship between WM impairment and disturbed functional connectivity both in the resting state and under various task. The connectivity could be modulated by the task demands. Some imagining studies reporting altered patterns of interregional functional connectivity in patients with schizophrenia during working memory task performance (Meyer-Lindenberg et al., 2001; Quintana et al., 2003; Schlosser et al., 2003a; Whalley et al., 2005). However, seldomly studies have investigate the functional connectivity in resting state. Connectivity within the DMN and FP have been found significantly different between resting state and 0-back, and was further modulated by memory load. (Grega Repovs 2012) Yet most of the existing find are during working memory tasks, we are still eager to find wheather we could detect functional connectivity impairments in resting state, which may be a potential trait marker of WM impairment in schizophrenia patients.

\section*{Materials and Methods}

\subsection*{Subjects}

The resting state fMRI data presented here was collected from six hospitals in China which participated in the Brainnetome Project for Schizophrenia. The six hospitals are Peking University Sixth Hospital (PKUH6); Beijing Huilongguan Hospital (HLG); Xijing Hospital (XJ); Henan Mental Hospital (HM);  Renmin Hospital of Wuhan University (RWU); and Zhumadian Psychiatric Hospital (ZMD). Henan Mental Hospital provided two distinct MRI scanners: Siemens (HMS) and General Electric (HMG), for a total of eight scanning centers. Patients and controls are matched as much as possible for age, sex, handedness, and race distributions within each site. The study at each center was approved by the local ethical review board. All the participants provided written informed consent.

All patients had a diagnosis of schizophrenia confirmed by trained psychiatrists using the Structured Clinical Interview for DSM-IV-TR Axis I Disorders (SCID-I/P). (First et al., 2002b). Exclusion criteria were a current neurologic disorder, a history of serious medical illness, substance dependence, pregnancy, electroconvulsive therapy within the last six months, or a diagnosis of any other Axis I disorder. The Positive and Negative Syndrome Scale (PANSS)  (Kay et al., 1989) was used to assess positive, negative, and general psychopathology symptoms in the patients. The healthy controls, who had no current axis I psychiatric disorders, were recruited from the local community near each center through advertisements. None of the HCs had any personal history of psychotic illness and no family history of psychosis in their first, second, or third degree relatives. All the participants were Han Chinese in origin, right-handed, and had no contraindications to MRI scanning. After extensive quality checking of the brain imaging data, 662 patients and 613 HCs were included in the analysis.
 
 
 
\subsection*{Data acquisition and preprocessing}
 
 
Two types of 3T MRI scanners (four Siemens, three General Electric) were used at the participating centers (details in table \#). To ensure equivalent data acquisition protocol and high quality imaging data, the scanning parameters of the functional scans at each of the six centers were set up by an experienced researcher before data acquisition. An echo planar imaging sequence was used to obtain the functional images, the parameters were as follows: 30 axial slices, TR = 2000 ms, TE = 30 ms, matrix = 64 × 64, flip angle = 90°, FOV = 240*240 mm 2, slice thickness = 4 mm, gap = 0.4 mm. A total of 250 brain volumes were collected, resulting in a total scan time of 500s. The MRI scan sequences and parameters for each center are listed in eTable 1. To be noticed, the time point number of the images from ZMD is 180 and is different from the 240 of the other sites. We did not exclude this site since we are eager to keep as much data and sites as possible, to support the validation of the multi-site analysis. The multi-site statisical stratergy we used is robust to various of volum numbers in different sites.    

%时间校正,头动校正,配准,降噪,滤波,空间平滑

The iamges were preprocessed with a based in-house software: Brainnetome Toolkit which utilized Statistical Parametric Mapping SPM8 (\url{http://www.fil.ion.ucl.ac.uk/spm}). The pipeline includes the following steps: The first ten images were deleted for the signal equilibration. The remaining iamges were conducted for slice acquisition correction and head motion correction.  The fMRI data which had
less than 3.0mm of head motion and 3.0° of angular rotation were included.  Moreover, the mean frame-wise displacement (FD) was computed by averaging FD\_i from every time point for each subject. There were no differences for the mean FD between groups ( t = 0.413, p = 0.682) (Table 1). Then the fMRI images were normalized to the standard Montreal Neurological Institute (MNI) template provided by SPM and resample to the 3-mm isotropic voxels. Artefacts due to changes in global, ventricle and white matter signals, residual motion were removed using voxel-wise regression. A temporal filter (0.01 Hz \textless f \textless 0.08 Hz) were used to reduce the low-frequency drift and physiological high frequency respiratory and cardiac noise. Finally, the data was smoothed with an isotropic Gaussian kernel of 6 mm full-width at half-maximum. 

Functional connectivity analysis

Regions of interest (ROIs)

The ROIs in this study were extracted from a previous meta analysis by Simon, in which they find regions that commonly acvite during various working memory tasks. A consistent and restricted "core" network emerged from conjunctions across analyses of specific task designs and contrasts: task effects for n-back and Sternberg tasks, verbal and non-verbal tasks, load effects and all three task components (encoding, maintenance, recall). These regions includs dosal area 44 , anterior insula, (pre-) SMA, and IPS, bilaterally. The peak coordinates are listed in table \#. A 5mm radis sphrere ROIs are created form these peak coordinates in MNI standard space.

\subsection*{fcMRI Analyses}

To perform the fcMRI analyses, time series from the resting-state scan were extracted by averaging the time series of all voxels in each ROI. The resulting time series of each ROI were then entered into the following connectivity analyses. The Pearson (full) correlation coefficient was used to estimate functional connectivity between each pair of time series. A Fisher Z transform was applied to transform from pearson r to z scores.

\subsection*{Statistical analysis}

Group differences in the basic demographics at each center were examined with two-tailed t-tests and chi-square tests using PASW Statistics 18.0.

For every single site separately, the diagnose effect on each FC value was assessed using a two-sample t-test comparing images between SZ and NC from each site. A second level random effects meta analysis was used to pooling the single site statistics, in order to assess the replication of the effects from the entire dataset. The effect sizes were calculated using Hedge’s g, which provides an unbiased standardized mean difference that incorporates a correction for small sample sizes. Hedge’s g was calculated using each center’s t-statistic and the corresponding sample size. All significance thresholds were set to p \textless 0.001 FDR-corrected for multiple comparison. The meta-analysis was conducted in R with meta-for packages.

We then performed association analysis for those fc connectivities that passed the significant test. We then examined the relationship between FC values of those that passed the significant test and demographic variables in patients.  Linear regression was then conducted to determine a possible association between those significantly FC connectivity and clinical scores including (PANSS positive , PANSS negative, PANSS general and PANSS total). These analyses are conducted both separately in each site and in the whole data. In whole data analysis, the site factor was included in the GLM model. An additional analyses were also performed with covariate of age and sex.

\section*{Results}

Demographics and clinical data

Sociodemographic and psychopathological data are presented by center in Table 1. No statistically significant differences in age and sex were noted between the schizophrenia patients and HCs at each center. The patients had significantly fewer years of education than the HCs in six out of eight centers.

Group comparisons of FC connectivities at each center

Mixed effect multicenter analysis

Compared to the normal group, four FC conncevities was reduced in the patients: they are the connection between the left IFG/caudal lateral prefrontal gyrus and the left anterior insula, Left IFG/caudal lateral prefrontal gyrus and the Left inferior frontal gyrus pars opercularis, Right intraparietal sulcus and the Left inferior frontal gyrus pars opercularis, Right intraparietal sulcus and the Right inferior frontal gyrus pars opercularis. No functional connectivity was detected with a enhanced effect in the patients group.  


Association analysis

We did not find any association between the FC and the clinical scores.
\section*{Discussion} 

\subsection*{Working memory networks}


Working memory is the result of various combinations of processes, no processes (and correspondingly no brain structures)are unique or specific to working memory( Eriksson, 2015). Many brain regions interact during working memory and include "executive" regions in the PFC, parietal cortex, and basal ganglia, as well as regions specialized for processing the particular representations to be maintained, such as the fusiform face area for maintaining face information. Persistent neural activity in various brain regions accompanies working memory and is functionally necessary for maintenance and integration of information in working memory.

The organization of human WM has long been the topic of psychological models (Atkinson and Shiffrin, 1968; Hebb, 1949), with maybe the most influential having been proposed by Baddeley and Hitch (1974). In Rottschy,2012, the WM network was identified by functional neuroimaging using quantiattive coordiante-based meta-analysis over almost 200 individual experiments. By pooling various working memory tasks, a main network was identified which mainly comprise the fronto-parietal network. By eliminating the effect of specific task designs and contrasts, a more restricted "core" network emerged from conjunctions analyses. A core network independently of the specific aspects and task features was identified using conjunction analysis. This network mainly comprise the dorsal area 44, anterior insula, (pre-)SMA and IPS. The dorsal areas 44/45 and the pre-SMA are part of the phonological loop, a subsystem that response for verbal working memory material maintenance. There has been evidence that the dorsal region of Broca's area is active only during the first part of the delay period, and is involved in the formation of an articulatory rehearsal program. Generally, the same brain regions dedicated to sensory processing are believed to store sensory information during delay periods and working-memory task performance. A conjunction of verbal vs. non-verbal material in Rottschy's study revealed that the BA44/45 area may also involed in non-verbal WM tasks, that it may not be a modality-specific area in working memory. The reduced functional connectivity may underpin the verbal working memory deficits in schizophrenia. The previous research has established that the PFC is causally involved in normal working  memory functioning. However, there is yet no consensus on the details of the functional organization of the PFC. The absent of DLPFC, which is believed to an neuro agent for the central executve module in Baddeley and Hitch's model in Rottschy's core network, may be due to some working memory task requires little manipulation process of the memory content. In fact, the lateral PFC clusters in the main network was subdivided to two, that the abstraction level of goals and task rules are suggested to peak in rostral PFC and decrease to the caudal part. The caudal LPFC register a working memory load effect while the rostral part was not. A meta analysis has find consistent activation in bilateral mid-ventrolateral prefrontal coretex or frontal operculum(BA45,47) whin N-back studies. (Owen, 2005) In this study, three subsets of the N-back tasks revealed similar activation pattern implicating prefrontal, premotor, and posterior parietal cortex, which suggested a core modality independent working memory network. The inhibitory processes apear to be mediated by area 45 (left lateral prefrontal structures) (Jonides,1998) in working memory tasks.  For  example,  Owen,Petrides and their colleagues [6,7] proposed that the mid-ventrolateral region (Brodmann’s area [BA] 45/47) supports the organization of response sequences based on information  retrieved  from  posterior  areas,  whereas  the mid-dorsolateral  region  (BA 9/46)  supports  the  active manipulation or monitoring of information within working memory(Patricia A Carpenter, 2000).  

The inferior frontal gyrus/anterior insula(IFG/AI) was suggested to be involved in elaborate attentional and working memory processing,  (Mattie Tops,2011). Some evidence has suggsted the IFG/AI might involved in cognitive control in working memory tasks. These ventral cortico-limbic control pathways that include the IFG/AI , may adapt to working memory context that differ in the level of predictability. Anohter within subject study find the left and right IFG showed a conjunction between working memory and inhibition tasks within subjects, which indicate some component of excutive function may interactivate with the working memory systerm in working memory tasks (MacNab, 2008). The fronto-opercular, intraparietal and anteriro cingulate cortex may form a circuit for non-articulatory maintenance of phonological information(Henseler 2009).  

The anterior insula is closely associated with working memory processes in healthy participants and shows gray matter reduction in schizophrenia (Clos, 2014). Another review give attention to the role of AI in switching between other large-scale networks to facilitate access to attention and working memory resources when a salient event is detected (Vinod Menon, 2010). One study has reveal the possiblility of IPS response for manipulation the information in verbal working memory, yet there has been not evidence in other modality. Yet this area is also involved in spatial working memory. The PPC is also considered to be involved in maintenance of information, however, the considerable variability in the location of the parietal peaks make it hard to have a conclusion.



\subsection*{Working memory impairment in Schizophrenia}


A large amount of studies focused on working memory in schizophrenia, most of these are by using task fmri.  The complex pattern of hyper- and hypoactivation consistently found across studies implies that rather than focusing on DLPFC dysregulation, researchers should consider the entire network of regions involved in a given task when making inferences about the biological mechanisms of schizophrenia (David C. Glahn, 2005). A PET study has found impaired interaction between right lateral prefrontal cortex and bilateral inferior parietal region in SZ patients compared with normal patients during working memory processing(Jae-Jin Kim, 2003). There has been research reporting the disconneciton between ...

\subsection*{Mega analyses}
As we considered, a single site imaging study may be biased by the scanner, imaging protocol, and clinical measures. To overcome these drawbacks, we acquired high quality fMRI data with common protocal across different sites from unique samples of schizophrenia patients with the same ethinic origin, as a potentially representative participant sample. The subjects are also recruited by the same critiria and the clinical scores are assessed by the stardards. Despite these control measures, the data and the statistical results may be influenced by the differences in psychopathology, exposuer to antipsoychotic medication andf the scanners used for iamge acquisitions across patients from different sites scanner type and some other potential effects. In order to assess the replication of the effects from the entire dataset within smaller subsets, Meta-analysis is used here to pooling resutls from single site and to increase the statistical power (J A. Turner, 2013, S G, Costafreda, 2009). To model/capture the heterogeneous induced by the different subjective recruitment strategies, cognitive paradigms, acquistion software and hardware, and the individial variance in coregistering to the template, we treat site factor as a random effect. By applying such multi-site experiment design and pooling stratergy, we were able to, first, reduce the possibility of biased results in a single site and provide a reliable and generalized results; second, extract novel insights from existing large-scale datasets by increasing the statistical power; thied ,the sample we collected is a representive righthanded Han chinese population which was collected from hospitals distributed across China.



\section*{Conclusions}

%% The Appendices part is started with the command \appendix;
%% appendix sections are then done as normal sections
%% \appendix

%% \section{}
%% \label{}

%% References
%%
%% Following citation commands can be used in the body text:
%% Usage of \cite is as follows:
%%   \citep{key}          ==>>  [#]
%%   \cite[chap. 2]{key} ==>>  [#, chap. 2]
%%   \citet{key}         ==>>  Author [#]

%% References with bibTeX database:

\section*{Acknowledgments}
%All visualizations were performed using ITK-SNAP%
%\footnote{
%http://www.itksnap.org/
%}
%\citep{Yushkevich2006} and
%DTI-TK.%
%\footnote{
%http://www.nitrc.org/projects/dtitk/
%}
%We also gratefully acknowledge Dr. Niels van Strien of the Norwegian University of Science and Technology
%who assisted in packaging the template construction algorithm in the very useful script \verb#buildtemplateparallel.sh#
%which is publicly available in ANTs.

\section*{References}

\bibliographystyle{elsarticle-harv}
\bibliography{references}


%% Authors are advised to submit their bibtex database files. They are
%% requested to list a bibtex style file in the manuscript if they do
%% not want to use model1-num-names.bst.

%% References without bibTeX database:

% \begin{thebibliography}{00}

%% \bibitem must have the following form:
%%   \bibitem{key}...
%%

% \bibitem{}

% \end{thebibliography}


\end{document}

%%
%% End of file `elsarticle-template-1-num.tex'.